\chapter{Einleitung}
\begin{singlespace}
\renewcommand*{\dictumwidth}{0.4\textwidth} 
\dictum[Richard Phillips Feynman \footnotemark]{It doesn't matter how beautiful your theory is, it doesn't matter how smart you are. If it doesn't agree with experiment, it's wrong.}
\footnotetext{1964, Transkipt der Messenger Lectures an der Cornell Universität: The Character of Physical Law}\bigskip 
\end{singlespace}

\section{Motivation/Relevanz/Hinführung zum Thema}

Aktuelle Unternehmen wie Nespresso, Netflix oder Uber machen deutlich, wie Geschäftsmodelle maßgeblich für den Erfolg eines Unternehmens verantwortlich sind. Die Fähigkeit zu innovativen, neuen Geschäftslogiken gilt bereits unlängst als Kernvoraussetzung für eine stetige Wettbewerbsfähigkeit von Unternehmen. Eine neue Studie von mehr als 3.000 CEOs propagiert den künftig steigenden Stellenwert innovativer Geschäftsmodelle (IBM Studie) um deren Marktfähigkeit weiterhin sicherzustellen. Verstärkt wird dieser Effekt durch die steigende Substituierbarkeit von Produkten und Dienstleistungen, hervorgerufen durch die fortschreitende Digitalisierung kommerzieller Güter. Dadurch reichen alleinige Produkt-, Dienstleistungs- und Prozessinnovationen nicht mehr aus um am Markt weiterhin wettbewerbsfähig zu bleiben. Bestätigt wird diese Vermutung durch empirische Studien, welche belegen, dass Unternehmen welche aktiv Geschäftsmodellinovationen betrieben haben nach 5 Jahren im Durchschnitt ihren Erfolg um sechs Prozent mehr steigern konnten als dies Wettbewerbern über Prozess- oder Produktinnovationen möglich war. Das Geschäftsmodell stellt sich daher in vielen Wirtschaftsbereichen künftig als alleiniges Differenzierungsmerkmal dar. Das durchbrechen von Branchenlogiken und kontinuierliche Anpassen der Geschäftsmodelle an die Umweltbedingungen stellen somit ein obligatorisches Mittel für Unternehmen dar, um auf Dauer konkurrenzfähig zu bleiben. Abbildung 2 verdeutlich den rasanten Anstieg der Literatur über Geschäftsmodelle und Geschäftsmodellinnovation.
Der Prozess der Geschäftsmodellinnovation zeichnet sich vor allem durch die Suche nach neuen Geschäftslogiken aus. So werden neue Wege erprobt, mit denen ein Unternehmen Wert generieren und festigen kann. Ferner beschreibt dieser Prozess das Erzeugen neuer sogenannter ’Nutzenversprechen’ gegenüber Kunden, Partnern und Lieferanten (Andreini, 2017, S. 4). Die Generierung und Implementierung neuer Geschäftsmodelle wird dabei durch Geschäftsmodellinnovation-Frameworks und Prozessen systematisch begleitet. Der Schritt der Erzeugung potentieller Geschäftsmodelle stellt sich dabei als kreative und gruppenbasierte Problemlösungsaufgabe dar. (vgl. Todd Lubat the createive process nach guilford). Dies verknüpft die erfolgreiche Geschäftsmodellentwicklung unmittelbar mit dem kreativen Schaffensprozess. Die Forderung der Gruppenunterstützung sowie der effektiven Unterstützung kreativer Arbeit kann durch Softwaretools positiv beeinflusst werden (siehe ebel et al)
Im Zeitalter steigender betrieblicher Unterstützung durch informationstechnische Anwendungssysteme und im Zuge des gestaltungsorientierten Ansatzes sowohl in der deutschen Wirtschaftsinformatik als auch der angloamerikanischen Disziplin des Information System (Design Science) werden daher Softwarelösungen vorgeschlagen, welche kreative Problemaufgaben bestmöglich unterstützen sollen. Diese sogenannten Creativity-Support-Systems stellen dabei vor dem Hintergrund der gestaltungsorientierten Wirtschaftsinformatik ein IT-Artefakt dar. Dieser Bereich der Wirtschaftsinformatik befasst sich dabei nicht nur ausschließlich mit der Entwicklung und Implementierung von IT Artefakten in Form von Softwareprototypen oder Anwendungssystemen, sondern trägt ebenfalls zur Evaluierung jener Artefakt bei. Die Artefaktevaluation untersucht dabei Prototypen hinsichtlich Ihrer Nützlichkeit sowie Auswirkungen konkreter Designentscheidungen in Anwendungssystemen. DeLone und McLean machten mit sehr stark in der Litertur zitierten ihrem Framework bereits deutlich/beschreiben die Notwendigkeit , dass IS-Systeme bzw. Artefaktimplementierungen hinsichtlich ihres effektiven Nutzen für Individuen als auch Unternehmen beurteilt/gemessen/untersucht werden müssen und bezeichnen IS-Sucess bzw. Effektivität/Nützlichkeit als DIE Dependent Variable in der Wirtschaft und spricht in diesem Zusammenhang bereits von kausalen Beziehungen einzelner Faktoren die die Qualität beeinflussen und empirisch untersucht werden sollten (variance model of sucess). In anbetracht komplexer Wechselwirkungne zwischen einzelnen Faktoren kritisierte er dabei oft fehlende kapselung bzw. Kontrolle gegenüber konkurirender Komponenten (zb. Einfluss von User-Satisfaction oder Systemqualität auf Unternehmenerfolg)... Neben semantischer Bedingheit und technischer Qualität von IS-Systemen macht das Effektivitätslevel dabei ein Großteil des wahrgeommenen IS-Erfolges aus. 
Zur Feststellung des Zusammenhanges zwischen dem Einsatz eines IT-Artefaktes und der gemessenen Nutzeneffektivität bei Anwendung stellen sich Experimente als geeignete Methodik dar. Frank bestätigte in einem Peer-Interview mit wichtigen Vertretern aus den Bereichen IS und WI, dass sich quantitativ, empirisch geleitete Forschung bereits als Gold Standard durchgesetzt hat. Ebenso macht Hess deutlich, dass für den Zweck der Artefaktevaluation experimentelle Forschungsmethoden als ideale Methodik darstellen um die geforderten kausalen Rückschlüsse von IT-Artefaktn auf deren Effektivität schließen zu können.
Trotz des Fokusses der deutschen WI auf den gestaltungsorientiren Ansatz und dem „ruf“ nach epmpirischen Methoden, insbesondere zur Artefaktevalaution, sind bislang sich bislang nur wenige … in der Literatur finden, welche den experimentellen Ansatz als rigorose Forschungsmethode wählten. [] macht hierbei deutlich, dass derzeit wenige Experimente als Vorlage genutzt werden können.
Neben Fallstudien und anderen Methoden nimmt das Feld bzw. Laborexperiment derzeit nur eine marginale Rolle ein (siehe bild wille).
Vor dem Hintergrund der Forschung zur Unterstützung von Geschäftsmodellenentwicklung und deren Unterstützung IT-Artefakte in Form von Anwendungen welche den CSS gebügen, richtet sich diese Arbeit an Forschern aus dem Bereich der Wirtschaftsinformatik und soll diesen einen Überblick geben, wie Experimente zur Artefakevaluation in der Literatur zum Themenschwerpunkt der Kreativität umgesetzt werden. Dabei sollen grundlegende Konzepte identifiziert und synthesisiert dargestellt wertden.: RQ1 Im Anschluss wird dann überprüft, ob die Erkenntisse der untersuchten Experimente berits Implikationen für die Forschung der GMMT bergen, welche so in der Literatur noch nicht umfassend kommuniziert wurden und Verbesserungpotential für den Aufbau dieser tools beinhalten.RQ2
Das Ziel dieser Arbeit ist es deshalb, einen Überblick darüber zu geben, wie in der deutschen als auch der internationalen Wirtschaftsinformatik (Information Systems) die Forschungsmethodik des Experiments eingesetzt wird um bestehende oder eingeführte IT-Artefakte zu evaluieren. Dabei werden allerdings nur jene Forschungen berücksichtigt, welche sich substantiell mit dem Konstrukt der Kreativität auseinandersetzten (RQ1).  
Durch diese Betrachtung wird dann im zweiten Schritt überprüft, ob sich aus den untersuchten Forschungen bereits Implikationen oder (…) für die Forschung zu Geschäftsmodellmodellierungstolls ergeben.
Im ersten Schritt werden Paper analysiert und innherlab von Bereichen miteiandner verglichen. In einem zweiten Schritt wird dann untersucht ob sich auffälligkeiten bei der Wahl bestimmter Methoden zeigen… Diese Auffälligkeiten werden dann im 3. Schritt dazu genutzt Handlungsempfehlen für den Bereich der GGMT abzuleiten. Diese Handlungsempfehlungen finden auf 2 Ebenen statt. Zum einen werden konkrete Vorschläge im Bereich der Implementierung unterbreit. Zum anderen werden Vorschläge gemacht, wie die Forschung (Artefaktevalauon) diesen Schritt durch Experimente optimal gestalten kann.

NEU: Geschäftsmodellinnovation ist wichtig (innovation neues differenzierungsmerkmal, wettbewerbsvorteile, CEO umfrage zeigt unbestrittene relevanz).  Die entwicklung neuer geschäftsmodelle stellt dabei eine kreative problemaufgabe dar weclhe zugleich auch als kollaborativ gilt, da ein solches organisationsziel die zusammenarbeit von personen verschiedener disziplinen erfordert. Aufgrund dieser relevanz sind Geschäftsmodelle sind ein großer Forschungsbereich in der wortschaftsinformatik (Veith eh al.). Ein teilgebeit der WI schreibt sich der gestaltungsorientierung zu, heißt Ziel ist es zweckmäßige IT-Artefakte (zb. software, aber auch modelle) zu erzeugen. Innerhalb des Designszyklus ist es die aufgabe, Teilnutzen von spezifischen Features und Funktionen zu ermitteln und deren Auswikung (sowohl positiv als auch negativ) empirisch korrekt zu belgen. Vacher et al beschreiben dazu in ihrem framework mehrere schritte, mit denen sich eine solche theoriebildung gestalten lässt. Es ist wichtig herauszufinden, welches teil eines artefakt welchen nutzen bringt oder auch nicht um daraus für folgende Iterationen im gestaltungszyklus lernen zu können (siehe auch shneidermann s.6). Meherere dem englischsprachigen IS zugehörigen autoren sehen diese ex-post evaluation ebenfalls als erforderlichen schritt. Eine rigorose vielversprechende method bildet in diesem rahmen das wissenschaftliche experiment.. Deren strenge kontrolle ermöglicht es beobachtete effekte kausal auf bestimmte softwarefeatures zurückzuführen.
Neben dieser solchen „Artefaktevaluatoin erfüllt das experiment im DSR ebenfalls eine weitere wichtige Funktion bei der Prüfung aufgestellter hypothesen, um diese in den fundus der knowledge base aufnehmen zu können ()rigoroses vorgehen nach hevner“.
Trotz der unbestrittenen relevanz des experiments, ist derzeit kein zusammenfassender überblick vorhanden, wie sich experimente im bereich der WI darstellen, welche sich haupsächlich mit dem thema kreativität befassen. Mit der Arbeit von müller existiert bereits ein Ansatz, jedoch ist dieser noch zu oberflächlich.
Um in zukunft die wahl passender experimente für bestimmte kreative features gestalten, soll diese arbeit einen überblick geben, wie sich experimente zum thema kreativitä in den letzten jahren gestatlet haben. Dazu wird im ersten schritt ein literatureview durchgeführt. Im zweiten schritt wird kontrolliert, ob sich daraus bereits konkrete implitationen für bmddt rgeben. Im werden theoretische Grundkonzepte des Experiments vorgestellt.

XXX beschreibt quantitaive methoden als gold standard. Quantitaive mehthoden beschreiben nach yyy verfahren, welche statistischxxxx dazu gehören auch experimente


\section{Problembeschreibung} \label{eqn:1}

Otroswki macht deutlich, dass es innerhalb der DSR nur wenig Übersicht darüber herrscth, wie die oftmals empfohlene Art der Evaluation durch Experimente statzufunden habe.

\section{Ziel der Arbeit}

Ziel der Arbeit ist es ein Überblick darüber zu geben, wie Autoren den experimentellen Forschungsansatz im Bereich der Kreativität einsetzten. Dabei werden Konzepte identifiziert, anhand denen sich ein Experiment beschreiben lässt. Folglich sollen der Überblick dazu genutzt werden Implikationen für BMDT abzuleiten.
Dabei liegt der Schwerpunkt der Untersuchung auf dem gestaltungsorientierten / Experimentdesign und weniger auf den statistischen Analysteil. Der Einsatz spezifischer Analysemethodiken wird oft bereits durch die Beschaffenheit des Experiments (Aufbau) fest vorgegeben (vgl. ANOVA, MANOVA Wahl bzgl. Anzahl der uV)


Es bestehen deruzeit noch Defizite in der Übertragung potentiellen WIssen aus der empirischen Forschung zu Kreativität.
Zur Lösung des Problems soll diese Arbeit 3 zusätzliche Beiträge zur Literatur bringen (siehe Bild \ref{fig:paper_year}):
\begin{itemize}
\item Schaffen eines Überblicks sowohl aktueller als auch vergangene Literatur im Bereich der Kreativitätsforschung zu Experimenten
\item Entwikclung einer taxonomotischen Darstellung verwendeter Methodn  der betrachteten Experimentalforschung und Identifizieren wiederkehrender Vorgehensweisen
\item Vergleich der Experimente hinsichtlich identifizierter Merkmale
\item Durch das Ableiten konkrter Handlungsempfehlungen für die Entwicklung künftiger Geschäftsmiodellentwiclungstools
\end{itemize}



\begin{figure}[h]
\centering
\caption{Untersuchsungsgegenstand der Arbeit, eigene Darstellung}
\resizebox{\textwidth}{!}{%
\begin{tikzpicture}
\tikzstyle{every node}=[font=\small]
\draw [fill=\fuellung, draw=none]  (-8.65,-0.3) node (v1) {} rectangle (9.6,-1.6);
\node[rechteck, minimum height=5cm, align=left, minimum width=4cm, text width=4.cm, draw, outer sep=10pt](4) at (-6.5,3) {
Konzept A\\
\qquad Subkonzept $A1$\\
\qquad Subkonzept $A2$\\
Konzept B\\
\qquad Subkonzept $B1$\\
\qquad Subkonzept $B2$\\
Konzept C\\
\qquad Subkonzept $C1$\\
\qquad Subkonzept $C2$\\
.\\.\\
Konzept $k$\\
\qquad Subkonzept $k1$\\
\qquad Subkonzept $k2$\\
};




\node[rechteck, minimum height=1cm, align=left, minimum width=4cm, text width=4.cm, draw=none, fill=none, outer sep=10pt] (5)at (-6.5,-1) {
Erkenntnisse\\
\qquad Positiver Einfluss \\
\qquad Negativer Einfluss \\
};

\node[rechteck, minimum height=1cm, align=left, minimum width=4cm, text width=18.cm, draw=none, outer sep=10pt] (6) at (0.5,-3) {
Implikationen für die Forschung im Bereich Geschäftsmodellmodellierungstools\\
\qquad Maßnahme $1$ \\
\qquad Maßnahme $2$ \\
\qquad $...$ \\
\qquad Maßnahme $n$ \\
};

\node[rechteck, minimum height=2.5cm, text height=.5cm,align=left, minimum width=4cm, text width=4cm, draw=none](3) at (-6.5,7.5) {
Konzept  \qquad\qquad  \\
\
};
\node [rechteck, minimum height=2cm, text height=.5cm,align=left, minimum width=14cm, text width=14cm, draw=none] (13)at (3,7.1) {
\begin{tabular}{llllllllllllllllll} 
{
\rotatebox[origin=c]{90}{ \cite{Paper [1]}} & \rotatebox[origin=c]{90}{ \cite{Paper [2]}} & \rotatebox[origin=c]{90}{ \cite{Paper [3]}} & \rotatebox[origin=c]{90}{ \cite{Paper [4]}} & \rotatebox[origin=c]{90}{ \cite{Paper [5]}} & \rotatebox[origin=c]{90}{ \cite{Paper [6]}} & \rotatebox[origin=c]{90}{ \cite{Paper [7]}} & \rotatebox[origin=c]{90}{ \cite{Paper [8]}} & \rotatebox[origin=c]{90}{ \cite{Paper [9]}} & \rotatebox[origin=c]{90}{ \cite{Paper [10]}} & \rotatebox[origin=c]{90}{ \cite{Paper [11]}} & \rotatebox[origin=c]{90}{ \cite{Paper [12]}} & \rotatebox[origin=c]{90}{ \cite{Paper [13]}} & \rotatebox[origin=c]{90}{ \cite{Paper [14]}} & \rotatebox[origin=c]{90}{ \cite{Paper [15]}} & \rotatebox[origin=c]{90}{ \cite{Paper [16]}} & \rotatebox[origin=c]{90}{ \cite{Paper [17]}} & \rotatebox[origin=c]{90}{ \cite{Paper [...]}}
}
\end{tabular}
};




\node[rotate=90, draw,align=center, draw=none] (1) at (-10.25,3) {Vergleichsdimensionen};

\node[rotate=270, draw,align=center,draw=none] (l) at (11.5,3) {RQ1\\};
\node[rotate=270, draw,align=center, draw=none] (r) at (11.5,-3) {RQ2\\};

\draw[] (1.south) -|  ++(.0,0) |-  node[fill=none, pos=.25, rotate=0, align=center] {}(4.north west);
\draw[] (1.south) -|  ++(.0,0) |-  node[fill=none, pos=.25, rotate=0, align=center] {}(4.south west);




\draw[pfeil_einfach] (5.west) -|  ++(-1.,0) |-  node[fill=none, pos=.25, rotate=90,above, align=center] {Synthese Implikationen \\für GMMT}(6.west);

\node[ rechteck, minimum height=6cm, align=left, minimum width=4cm, text width=13.cm, draw=none, outer sep=10pt] (inner) at (2.95,3) {
};
\draw[] (l.south) -|  ++(-.5,0) |-  node[fill=none, pos=.25, rotate=0, align=center] {}(inner.north east);
\draw[] (l.south) -|  ++(-.5,0) |-  node[fill=none, pos=.25, rotate=0, align=center] {}(inner.south east);

\draw[] (r.south) -|  ++(-.5,0) |-  node[fill=none, pos=.25, rotate=0, align=center] {}(6.north east);
\draw[] (r.south) -|  ++(-.5,0) |-  node[fill=none, pos=.25, rotate=0, align=center] {}(6.south east);

\draw[draw](9.6,6.2) -- (-8.5,6.2);
\draw[draw](9.6,8.5) -- (-8.5,8.5);
\draw[draw](9.6,-4) -- (-8.5,-4);
\draw[draw](9.6,-2) -- (-8.5,-2);
%\draw[draw](9.6,6.26) -- (-8.4,6.26);
\draw [fill=mygray!50!, draw=none, align=center]  (-4,6) node (v1) {} rectangle node{Konzeptorientierter Vergleich der Experimente \\in Form einer Taxonomie} (9.6,0);
\end{tikzpicture}
}

\label{fig:myLabel}
\end{figure}

\section{Forschungsvorgehen/Methodik}




\section{Gang der Arbeit}
Die Arbeit im allgmeinen folgt dem qualitativen Forschungsprozess und einem rein deduktiven Vorgehen. Dabei werden aus generellen/allgemeiem Wissen (Literaturreview zu Experimenten zum Thema Kreativität im Bereich der WI) konkrete Design-Hypothesen für Geschäftsmodellierungstools abgeleitet. Der deduktive Prozess ist dabei angelehnt am kommunizierten Vorgehesnmodell nach Spens/Kova.
Zunächst wird in Kapitel 2 die zur Beantwortung der Forschungsfragen relevanten Theorien und Hintergrundinformationen dargestellt. Im nächsten Kapitel werden dann bereits vorhandene Experimentalliteratur durchsucht und aufbereiter ("Exisitn theoretical knowledge"). Aus dem dadruch gewonnen theorethischen Rahmenwerk werden dann konkrete Hypothesen für die Forschung der Geschäftsmoellierungstools abgleietet. Die in dieser Arbeit durchgeführten Prozessschritte sind in Abbildung 1 visualisiert. 
Der logische Aufbau der Arbeit orientiert sich am deduktiven Forschungsprozess, dargestellt von Kopac und Spens. 

\missingfigure{Bild Gang der Untersuchung}
\noindent
\begin{table}[H]
\centering
\begin{subfigure}[t]{0.45\linewidth}
\begin{tabular}{@{}c@{}}
\resizebox{1\linewidth}{!}{% standard .82
\begin{tikzpicture}[every node/.style={anchor=center},>=stealth]
     \matrix (tree) [%
       matrix of nodes,
       minimum size=1cm,
       column sep=1cm,
       row sep=.5cm,
       ampersand replacement=\&,
       row 1/.style={nodes={draw,top color=white, bottom color=black!20,rounded corners, text width=2cm, align=center, outer sep=.1cm}},
       row 3/.style={nodes={draw,top color=white, bottom color=black!20,rounded corners, text width=2cm, align=center, outer sep=.1cm}},
       ]
     {
         \& {Domäne \\ } \& \\
         \&\quad\&\\
     Field  \& \&  Person\\
     };

%      \draw[->] (tree-1-2) -- (tree-3-1) node [midway,above] {$p$};
%      \draw[->] (tree-1-2) -- (tree-3-3) node [midway,below] {$1-p$};
%     \draw[] (tree-3-1) -- (tree-3-3) node [midway,below] {$1-p$};
%      \draw[->] (tree-3-3) -- (tree-3-1) node [midway,below] {$1-p$};
\draw (tree-1-2) to [bend left=0] node [above, fill=none, sloped] {informiert} (tree-3-3)  ;
\draw (tree-1-2) to [bend left=0]  node [above, sloped] {übernimmt} (tree-3-1);
\draw (tree-3-1) to [bend left=5]  node [above] {überträgt} (tree-3-3);
\draw (tree-3-3) to [bend left=5]  node [below] {bewertet} (tree-3-1);

\end{tikzpicture}
}





\end{tabular}
\subcaption{Allgemeines Systemmodell}
\end{subfigure}
\hfill
\begin{subfigure}[t]{0.45\linewidth}
\begin{tabular}{@{}c@{}}
\resizebox{1\linewidth}{!}{% standard .82
\begin{tikzpicture}[every node/.style={anchor=center},>=stealth]
     \matrix (tree) [%
       matrix of nodes,
       minimum size=1cm,
       column sep=1cm,
       row sep=.5cm,
       ampersand replacement=\&,
       row 1/.style={nodes={draw,top color=white, bottom color=black!20,rounded corners, text width=2cm, align=center, outer sep=.1cm}},
       row 3/.style={nodes={draw,top color=white, bottom color=black!20,rounded corners, text width=2cm, align=center, outer sep=.1cm}},
       ]
     {
         \& {Domäne \\ } \& \\
         \&\quad\&\\
     Field  \& \&  Person\\
     };

%      \draw[->] (tree-1-2) -- (tree-3-1) node [midway,above] {$p$};
%      \draw[->] (tree-1-2) -- (tree-3-3) node [midway,below] {$1-p$};
%     \draw[] (tree-3-1) -- (tree-3-3) node [midway,below] {$1-p$};
%      \draw[->] (tree-3-3) -- (tree-3-1) node [midway,below] {$1-p$};
\draw (tree-1-2) to [bend left=0] node [above, fill=none, sloped] {informiert} (tree-3-3)  ;
\draw (tree-1-2) to [bend left=0]  node [above, sloped] {übernimmt} (tree-3-1);
\draw (tree-3-1) to [bend left=5]  node [above] {überträgt} (tree-3-3);
\draw (tree-3-3) to [bend left=5]  node [below] {bewertet} (tree-3-1);

\end{tikzpicture}
}





\end{tabular}
\subcaption{Systemmodell im Kontext von GMI}
\end{subfigure}
\caption{ Darstellung des Systemmodells nach Csik und Anwendung auf die Entwicklung neuer Geschäftsmodelle}
\end{table}




\begin{figure}[t]
\centering
\resizebox{\textwidth}{!}{%
\begin{tikzpicture}
[auto,
decision/.style={diamond, draw=blue, thick, fill=blue!20,
text width=4.5em,align=flush center,
inner sep=1pt},
block/.style ={ rectangle, draw, fill=none,  text centered, minimum height=15mm, text width=12em},
line/.style ={draw, -latex,shorten >=2pt},
cloud/.style ={draw=red, thick, ellipse,fill=red!20,
minimum height=2em}]
\matrix [column sep=4cm,row sep=13mm, ampersand replacement=\&]
{
% row 1
 \node [block, draw=none, minimum height=1cm] (headerl) {Kapitelstruktur der Arbeit};\&
 \node [block, draw=none, minimum height=1cm] (headerr) {Deduktiver Forschungsprozess nach Spens et. al};  \\
% row 2
\node [block] (kap1) {Kapitel 1}; \& 
\node [block] (step1) {Prior theoretical knowledge}; \\
% row 3
\node [block] (kap2) {Kapitel 2}; \&
\node [block] (step2) {Theoretical framework}; \&  \node [cloud] (rq1) {RQ1}; \\
% row 4
\node [block] (kap3) {Kapitel 3}; \& \node [block] (step3) {Suggestion of HP};   \\
% row 5
\node [block] (kap4) {Kapitel 5};\& \node [block] (step4) {Testing of HP \\ (e.g. Experiment)};  \& \node [decision] (exp) {Verifi-kation?};\\
% row 6
\node [block] (kap5) {Kapitel 6}; \& \node [block] (step5) {New knowledge};  \\
};

% \path[draw] (init) -- (identify);
% \path (identify) -- (evaluate);
% \path (evaluate) -- (decide);
% \path (update) |- (identify);
% \path (decide) -| node [near start] {yes} (update);
% \path (decide) -- node [midway] {no} (stop);
% \path [dashed] (expert) -- (init);
% \path [dashed] (system) -- (init);
% \path [dashed] (system) |- (evaluate);
%\draw[red] (headerr) to [bend left=0] node [] {$sdsd$} (step1)  ;
\draw [line](kap1.east) to [bend left=0] node [above] {$p$} (step1.west)  ;
\draw [line](kap2.east) to [bend left=0] node [above] {$p$} (step1.west)  ;
%\draw [line, red](step2.east) to [bend left] node [right] {$$} (step5.east)  ;
% \draw (tree-3-1.east) to [bend left=0] node [above] {$p$} (tree-2-2.west)  ;
% \draw (tree-4-1.east) to [bend left=0] node [above] {$p$} (tree-2-2.west)  ;
% \draw (tree-5-1.east) to [bend left=0] node [above] {$p$} (tree-2-2.west)  ;
% \draw (tree-5-1.east) to [bend left=0] node [above] {$p$} (tree-4-2.west)  ;
% \draw [red](tree-2-2.east) to [bend left] node [right] {$$} (tree-6-2.east)  ;
\draw[line] (step1) to [bend left=0] node [] {$p$} (step2)  ;
\draw[line] (step2) to [bend left=0] node [] {$p$} (step3)  ;
\draw [line](step3) to [bend left=0] node [] {$p$} (step4)  ;
\draw[line, dashed] (exp.north) -|  ++(0,.5) -|  node [above, pos=.2, align=center] {Nein} node [below, pos=.2, align=center] {ggf. Anpassung der HP} (step4.north)  ;
\draw[line] (exp.south) |-  node [above, pos=.75] {Ja/Nein} (step5.east)  ;
%\draw[line] (step4) to [bend left=0] node [] {$p$} (step5)  ;
\draw[line] (step4) to [bend left=0] node [] {} (exp)  ;
\draw[line] (step2) to [bend left=0] node [] {} (rq1)  ;

\end{tikzpicture}}
\caption{Darstellung der Vorgehensweise in der Arbeit: Kapitelstruktur links, Deduktiver Forschungsprozess rechts (in Anlehnung an Spens et al (modifiziert))}
\end{figure}


\begin{figure}[t]
\centering
\resizebox{\textwidth}{!}{%
\begin{tikzpicture}
[auto,
decision/.style={diamond, draw=blue, thick, fill=blue!20,text width=4.5em,align=flush center,inner sep=1pt},
block/.style ={ rectangle, draw, fill=white,  text centered, minimum height=15mm, text width=12em},
line/.style ={draw, -latex,shorten >=2pt},
cloud/.style ={draw=red, thick, ellipse,fill=red!20, minimum height=3em, text width=5em, align=center}]

\draw [fill=black!40, fill opacity=.2, draw=none, shading=Mandelbrot set] (6.6,7.4) rectangle (11.2,-8.1);
\draw [fill=black!40, fill opacity=.2, draw=none] (-5.4,7.4) rectangle (5.3,-8.1);
\draw [fill=black!40, fill opacity=.2, draw=none] (-11.3,7.4) rectangle (-6.6,-8.1);
%\draw  [dashed](-0.2,8.8) --(-0.2,-8.2);
\matrix (table) [column sep=1.5cm,row sep=1.2cm, ampersand replacement=\&,  nodes in empty cells]
{
% row 1
 \node [block, draw, minimum height=2.3cm, fill=none] (headerll) {\textbf{Kapitelstruktur der Arbeit}};\&
 \node [block, draw=none, minimum height=1cm, fill=none] (headerl) {\\ \quad\\\quad\\\quad\\Theorie}; \&
 \node [block, draw=none, minimum height=1cm, fill=none] (headerr) {\\ \quad\\\quad\\\quad\\Empirie};  \&
 \node [block, draw=, minimum height=2.3cm, fill=none] (headerrr) {\textbf{Ergebnisse}};   \\

% row 2
\node [block] (kap1) {Kapitel 1\\Problembeschreibung und Zielsetzung};
\&  \node [block, draw=none, fill=none] (22) {};
\&  \node [] (23) {};
\&  \node [] (24) {};
\\

% row 3
\node [block] (kap2) {Kapitel 2\\Grundlagen und Definitionen}; \& 
\node [block, red, fill=white] (step1) {Prior theoretical knowledge};  
\&  \node [] (33) {};
\&  \node [] (34) {};
\\
% row 4
\node [block] (kap3) {Kapitel 3\\Systematische Literaturanalyse}; \&
\node [block, red, fill=white] (step2) {Theoretical framework}; \&   

 \&\node [cloud, align=center] (rq1) {RQ1};\\
% row 5
\node [block] (kap4) {Kapitel 4\\Implikationen für GM-Modellierungstools}; \& 
\node [block, red, fill=white] (step3) {Suggestion of HP}; \&  
 \&\node [cloud] (rq2) {RQ2}; \\
 
% row 6
\node [block] (kap5) {Kapitel 5\\Diskussion der Ergebnisse};\& 
 \& \node [block, blue,  fill=white] (step4) {Testing of HP \\ (e.g. Experiment)};
 \&  \node [cloud, fill=blue!20, draw=blue] (artefakt) {Artefakt-evaluation};\\
 
% row 7
\node [block] (kap6) {Kapitel 6\\Fazit und künftige Forschungsfelder}; \& 
\node [block, blue,  fill=white] (step5) {New knowledge};  
\&  \node [block, fill=none, draw=none] (73) {};
\&  \node [] (test4) {};\\

% row 8
\node [block, draw=red] (kap7) {}; \& 
\node [block, draw=blue] (step6) {};  
\& 
\&  \node [] (test4) {};\\
};



 \node[fit=(headerl)(headerr), draw, minimum height=1cm, fill=none]{\textbf{Deduktiver Forschungsprozess\\ nach Spens et. al}};
 
 % \node[fit=(22)(73), draw, minimum height=1cm, fill=black!20, fill opacity=.8]{Deduktiver Forschungsprolllzess\\ nach Spens et. al};
 

 
%Pfeile Kapitelstruktur
\draw [line](kap1) to [bend left=0] node [above] {} (kap2)  ;
\draw [line](kap2) to [bend left=0] node [above] {} (kap3)  ;
\draw [line](kap3) to [bend left=0] node [above] {} (kap4)  ;
\draw [line](kap4) to [bend left=0] node [above] {} (kap5)  ;
\draw [line](kap5) to [bend left=0] node [above] {} (kap6)  ;
\draw [line, red](kap7.west) to [bend left=0] node [above] {durchgeführte Schritte} (kap7.east)  ;
\draw [line, dashed, blue](step6.west) to [bend left=0] node [above] {geplante Schritte} (step6.east)  ;

%Pfeile Interaktion Kapitel zu Prozess
\draw [line, blue, dashed](step4.east) to [bend left=0] node [above] {$p$} (artefakt.west)  ;
\draw[line,dashed, blue] (artefakt.south) |-  ++(0,0) |- (step5.east)  ;
\draw[line,dashed] (kap6.east) |-  ++(.7,0) |- (step4.west)  ;
\draw [line](kap2.east) to [bend left=0] node [above] {$p$} (step1.west)  ;
\draw [line, dashed](kap2.east) to [bend left=0] node [above] {$p$} (step2.west)  ;
\draw [line](kap3.east) to [bend left=0] node [above] {$p$} (step2.west)  ;
\draw [line](kap4.east) to [bend left=0] node [above] {$p$} (step3.west)  ;
\draw[line, red] (step1) to [bend left=0] node [] {$p$} (step2)  ;
\draw[line, red] (step2) to [bend left=0] node [] {$p$} (step3)  ;
%\draw [line](step3.south) to [bend left=0] node [] {$$} (step4.north)  ;
%\draw[line, dashed] (exp.north) -|  ++(0,.5) |-  node [above, pos=.75, align=center] {Nein} node [below, pos=.75, align=center] {ggf. Anpassung der HP} (step3.east)  ;
%\draw[line] (exp.south) |-  node [above, pos=.75] {Ja/Nein} (step5.east)  ;
\draw[line, blue] (step3.south) -|  ++(0,-.5) -|  (step4.north)  ;
\draw[line, blue] (step4.south) -|  ++(0,-.5) -|  (step5.north)  ;
%\draw[line] (step4.south) to [bend left=0] node [] {$p$} (step5.north)  ;
\draw[line, red] (step2) to [bend left=0] node [] {} (rq1)  ;
\draw[line, red] (step3) to [bend left=0] node [] {} (rq2)  ;




\end{tikzpicture}
}
\caption{Darstellung der Vorgehensweise in der Arbeit: Kapitelstruktur links, Deduktiver Forschungsprozess rechts (in Anlehnung an Spens et al (modifiziert))}
\end{figure}


\begin{figure}[t]
\centering
\resizebox{\textwidth}{!}{%
\begin{tikzpicture}
[auto, 
decision/.style={diamond, draw=blue, thick, fill=blue!20,text width=4.5em,align=flush center,inner sep=1pt},
block/.style ={ rectangle, draw, fill=white,  text centered, minimum height=15mm, text width=12em},
block_todo/.style ={ rectangle, draw, fill=white,  text centered, minimum height=15mm, text width=12em, rounded corners, outer sep=5},
block_done/.style ={ rectangle, draw, fill=white,  text centered, minimum height=15mm, text width=12em, rounded corners, outer sep=5},
block_normal/.style ={ rectangle, draw, fill=white,  text centered, minimum height=15mm, text width=12em, rounded corners, outer sep=5},
block_phantom/.style ={ block_normal, draw=none},
block_background/.style ={ rectangle,  draw=, fill=none,  text centered, minimum height=15mm,fill opacity=.1, rounded corners},
output/.style ={trapezium,draw,fill=none, minimum height=10mm,  align=center, trapezium left angle=60, trapezium right angle=120, text width=70},
line_inner/.style ={draw, -latex,shorten >=-4pt, shorten <=-4pt},
line_outer/.style ={draw, -latex,shorten >=4pt, shorten <=4pt}]
]


\matrix (table) [column sep=1.5cm,row sep=1cm, ampersand replacement=\&,  nodes in empty cells, in front of path]
{
% row 1
\node [block_normal, draw=none] (11) {\textbf{Logische Struktur der Arbeit}};\& [.6cm]
 \node [block_normal, fill=none, draw=none] (12) {\\ \quad\\\quad\\\quad\\\quad\\\quad\\\textbf{Empirie}}; \&[-1cm]
 \node [block_normal, fill=none, draw=none] (13) {\\ \quad\\\quad\\\quad\\\quad\\\quad\\\textbf{Theorie}};  \& [0cm]
 \node [] (14) {};   \\[-.5cm]

% row 2
\node [block_normal] (21) {Kapitel 1\\Problembeschreibung und Zielsetzung};
\&  \node [block_done] (22) {(1) Theoretisches Vorwissen};
\&  \node [block_phantom] (23) {};
\&  \node [] (24) {};
\\

% row 3
\node [block_normal] (31) {Kapitel 2\\Grundlagen und Definitionen}; \& 
\node [block_done] (32) {(2) Theoretisches Rahmenwerk};  
\&\node [block_phantom] (33) {};  ;
\& \node [output] (34) {RQ1\vphantom{\begin{tabular}{c}x\\ y\end{tabular}}};
\\
% row 4
\node [block_normal] (41) {Kapitel 3\\Systematische Literaturanalyse}; \&
\node [block_done] (42) {(3) Ableiten von Hypothesen}; \&   
\node [block_phantom] (43) {}; 
 \&\node [output] (44) {RQ2\vphantom{\begin{tabular}{c}x\\ y\end{tabular}}};\\
 
% row 5
\node [block_normal] (51) {Kapitel 4\\Implikationen für GM-Modellierungstools}; \& 
\node [block_phantom] (52) {}; \&  
\node [block_done] (53) {(4) Prüfung der Hypothesen\\ (z.b. experimentell)};
 \&\node [output] (54) {Zukünftige Forschungsarbeiten}; \\

% row 6
\node [block_normal] (61) {Kapitel 5\\Diskussion der Ergebnisse};\& 
\node [block_done] (62) {(5) Finale Schlussfolgerung / \\Neues Wissen};
 \& \node [block_phantom] (63) {};
 \&  \node [] (64) {};\\
 
% row 7
\node [block_normal] (71) {Kapitel 6\\Fazit und künftige Forschungsfelder}; \& 
\node [block, fill=none, draw=none] (72) {};  \&  
\node [block, fill=none, draw=none] (73) {};\&  
\node [] (74) {};\\
};


\begin{scope}[on background layer]
%%%HEADER%%%%%
\node[fit=(12)(13), draw=none, minimum height=1cm, fill=none]{\textbf{Deduktiver Forschungsprozess\\ nach Spens et. al}};
%%%%%SPALTE1%%%%%%%%%
\node[fit=(21)(71), block_background]{};
%%%%%SPALTE2%%%%%%%%
\node[fit=(22)(63), block_background]{};
\end{scope}
 
 
%Pfeile Kapitelstruktur
 \draw [line_inner](21) to [bend left=0] node [above] {} (31)  ;
 \draw [line_inner](31) to [bend left=0] node [above] {} (41)  ;
 \draw [line_inner](41) to [bend left=0] node [above] {} (51)  ;
 \draw [line_inner](51) to [bend left=0] node [above] {} (61)  ;
 \draw [line_inner](61) to [bend left=0] node [above] {} (71)  ;


%Pfeile Prozess
 \draw [line_inner](22) to [bend left=0] node [above] {} (32);
 \draw [line_inner](32) to [bend left=0] node [above] {} (42)  ;
 \draw [line_inner](42.south) to [bend left=0] node [above] {} (53)  ;
 \draw [line_inner](53) to [bend left=0] node [above] {} (62.north)  ;

%Pfeile Interaktion Kapitel Prozess
 \draw [line_outer](21.east) to [bend left=0] node [above] {} (22.west)  ;
% \draw [line_outer](31.east) to [bend left=0] node [above] {$$} (22.west)  ;
  \draw [line_outer](31.east) -| ++ (.5,0) --  node [above] {} (22.west)  ;
% \draw [line_outer](41.east) to [bend left=0] node [above] {$$} (32.west)  ;
 \draw [line_outer](41.east) -| ++ (.5,0) --  node [above] {} (32.west)  ;
 %\draw [line_outer](51.east) to [bend left=0] node [above] {$$} (42.west)  ;
  \draw [line_outer](51.east) -| ++ (.5,0) --  node [above] {} (42.west)  ;
 %\draw [line](61) to [bend left=0] node [above] {$$} (22)  ;
% \draw [line](71.east) to [bend right=40] node [above] {$$} (53)  ;
 \draw [line_outer, dashed,shorten >=-3pt](71.east) -|   node [above, pos=.3] {Theorieprüfung in Anlehung }  node [below, pos=.3] {an Kuechler et al.}(53)  ;
 
%Pfeile Output
 \draw [line_outer, shorten >=1pt](33) to [bend left=0] node [above] {} (34)  ;
 \draw [line_outer,shorten >=1pt](43) to [bend left=0] node [above] {} (44)  ;
 \draw [line_outer, dashed,shorten <=1pt](54) to [bend left=0] node [above] {} (53)  ;
  \draw [line_outer, dashed](54) |-  node [above] {} (63)  ;

\end{tikzpicture}
}
\caption{Darstellung der Vorgehensweise in der Arbeit: Kapitelstruktur links, Deduktiver Forschungsprozess rechts (in Anlehnung an Spens et al (modifiziert))}
\end{figure}



\begin{figure}[t]
\centering
\resizebox{\textwidth}{!}{%
\begin{venndiagram3sets}[labelOnlyA={},labelOnlyB={},labelOnlyC={},labelABC={$x$},labelNotABC={$N$=59},tikzoptions={scale=1.9,thick}]
\fillACapBCapC
\setpostvennhook
{
\draw[<-] (labelA) -- ++(135:2cm) node[above, align=center,text width=5cm] {Wirtschaftsinformatik: Betrachtung von Softwareartefakten};
\draw[<-] (labelB) -- ++(45:2cm) node[above, align=center,text width=5cm] {Kreativität: Generierung bzw. Bewertung von Ideen};
\draw[<-] (labelC) -- ++(-90:2cm) node[below, align=center,text width=10cm] {Experiment: Vorhandene Randomisierung und Kontrollgruppendesign};
\draw[<-] (labelABC) -- ++(0:3cm) node[right,text width=2cm,align=flush right] {Zielliteratur};
\draw[<-] (labelNotABC) -- ++(-135:1cm)
node[below,text width=5cm,align=flush left]
{Identifizierte und analysierte Paper};
}


\end{venndiagram3sets}
}
\caption{Darstellung der Kriterien zur Identifikation relevanter Arbeiten als Anfangsschritt der systematischen Litertaurrecherche}
\end{figure}