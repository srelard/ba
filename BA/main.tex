%//Allgemeine Formatierung\\%
\documentclass[
%DIV = calc, %automatische Berechnung vonsseitenrändern
10pt,a4paper,oneside,bibtotoc, liststotoc, headsepline, smallheadings, openright, fleqn,appendixprefix, BCOR5mm]
{scrbook}
\usepackage[utf8]{inputenc}
\usepackage[ngerman]{babel}
\usepackage[T1]{fontenc}
\usepackage{amsmath}
\usepackage{amsfonts}
\usepackage{amssymb}
\usepackage{amsthm}
\usepackage{exscale} % Anpassung mathematischer Symbole an die Schriftgröße
\usepackage{amstext} % \text in mathematischer Umgebung
\usepackage{verbatim} % Funktion zum Schreiben von unformatiertem Text
\usepackage{float} % Ermöglicht das Erstellen von eigenen Float Umgebungen
\usepackage[section]{placeins} % Verhindert das Wandern von floats in eine andere section
\usepackage{makeidx}

%//Syntaktische Formatierung\\%
\usepackage{scrlayer-scrpage}%Kopf- Fußzeile
\usepackage[onehalfspacing]{setspace} % Einstellen des Zeilenabstandes
\singlespacing % Zeilenabstand setzen (alternativ: onehalfspacing, doublespacing)
\usepackage[
includeheadfoot,
left=3cm,right=2cm,top=2cm,bottom=2cm, showframe]{geometry}
\setcounter{secnumdepth}{3}
\setcounter{tocdepth}{4}

%//Font\\%
\usepackage{lmodern}

%//Captions, Bilder\\%
\usepackage{subcaption} % Anzeigen von Bildern aus mehreren Einzelbildern
\captionsetup[subfigure]{list=false, font=large, labelfont=bf,labelformat=brace, position=top}

%//Diverses\\%
\renewcommand{\topfraction}{0.85} % Anteil einer Seite, die von Floats belegt sein darf (default 0.7)
\renewcommand{\textfraction}{0.1} % Anteil einer Seite, die mindestens Text sein muss (default 0.2)
\renewcommand{\floatpagefraction}{0.75} % Anteil einer Seite, die ein Float einnehmen darf, ohne auf die nächste Seite gesetzt zu werden (default 0.5)
\usepackage{array} % Tabellen
\usepackage{rotating} % rotierte Tabellen und Bilder
\usepackage{xcolor} 
\usepackage{graphicx}
\newtheorem{theorem}{Satz}
\newtheorem{definition}{Definition}
\newtheorem{lemma}{Lemma}
\usepackage[withpage]{acronym}

%//Tikz etc.\\%
\usepackage{tikz}
\usetikzlibrary{arrows}
\usetikzlibrary{shapes.geometric}
\usepackage{pgfplots}
\usepackage{pgfplotstable}
\usepgfplotslibrary{dateplot}
\usepgfplotslibrary{units}

%//Tabellen etc.\\%
\usepackage{booktabs}
\usepackage{multirow}
\usepackage{tablefootnote}
\newcolumntype{L}[1]{>{\raggedright\arraybackslash}p{#1}} % linksbündig mit Breitenangabe
\newcolumntype{C}[1]{>{\centering\arraybackslash}p{#1}} % zentriert mit Breitenangabe
\newcolumntype{R}[1]{>{\raggedleft\arraybackslash}p{#1}} % rechtsbündig mit Breitenangabe
%\newcolumntype{C}[1]{>{\centering\arraybackslash}m{#1}} vertikale zentrierung

%//Zitate und Links\\%
\usepackage{csquotes}
\usepackage[]{hyperref}
\hypersetup{
bookmarksopen=true,
pdfpagelabels=true,
plainpages=false,
colorlinks=true,
citecolor=blue,
%linkcolor=blue,
urlcolor=blue}

%//BibLatex\\%
\usepackage[backend=biber, 
style=authoryear, %% Zitierstil
natbib=true, %% Bereitstellen von natbib-kompatiblen Zitierkommandos
hyperref=true, %% hyperref-Paket verwenden, um Links zu erstellen
maxcitenames=1,
uniquelist=false,
bibstyle=numeric,
url=false,
doi=false
]{biblatex}
%\addbibresource{bibfile.bib} %% Einbinden der bib-Datei
%\DefineBibliographyStrings{ngerman}{andothers={et al.}} % et al. statt u.a.
%\renewcommand*{\nameyeardelim}{\addcomma\space} % Macht im Zitat ein Komma vor das Jahr

%//Metadaten\\%
\newcommand{\logo}{
	\includegraphics[width=0.6\textwidth]{images/Logo_Uni_Paderborn}\\
}
\newcommand{\fachbereich}{
	\begin{tabular}{c}
	Fachgebiet Wirtschaftsinformatik   \\
	Information Management \& E-Finance  \\
	Prof. Dr. Dennis Kundisch    \\
	Universität Paderborn   \\
	\end{tabular}
}
\newcommand{\titel}{
	Experimente in der Wirtschaftsinformatik zu Kreativität: Ein
	Systematischer Literaturüberblick und Implikationen für
	Geschäftsmodellmodellierungstools
}
\newcommand{\vorgelegtvon}{
	vorgelegt im Rahmen der Abschlussprüfung\\
	für den Bachelor-Studiengang\\
	der Wirtschaftsinformatik
}
\newcommand{\autorpruefer}{
	\begin{tabular}{ll}
      Autor: & Sebastian Relard \\
      Matr.-Nr.: & 6697977\\
      Adresse: & Borchener Straße 106\\
      & 33098 Paderborn\\
      E-Mail: & srelard@mail.upb.de\\
      &\\
      &\\
      Erstprüfer: & Prof. Dr. Dennis Kundisch \\
      Zweitprüfer: & Prof. Dr. Leena Suhl \\
      Betreuer: & Daniel Szopinski \\
	\end{tabular}
}
\newcommand{\datum}{
	Paderborn, den \today
}



\begin{document}
\begin{onehalfspace} %Start 1,5-facher Zeilenabstand
\pagestyle{empty}
\pagenumbering{alph}
\begin{titlepage}
\singlespacing
\centering

\logo
\vspace{1cm}

\fachbereich \par
\vspace{2cm}

{\scshape\huge Bachelorarbeit \par}
\vspace{1.5cm}

{\huge\bfseries \titel \par}
\vspace{1.7cm}

\vorgelegtvon \par
\vspace{2.cm}

{\autorpruefer \par}
\vfill

\datum
\end{titlepage} \cleardoublepage
\noindent
Hiermit erkläre ich an Eides Statt, dass ich die vorliegende Arbeit selbständig und ohne unerlaubte fremde Hilfe angefertigt, andere als die angegebenen Quellen und Hilfsmittel nicht benutzt und die den benutzten Quellen und Hilfsmitteln wörtlich oder inhaltlich entnommenen Stellen als solche kenntlich gemacht habe.\\
\\
\\
Paderborn, \today
\begin{flushright}
------------------------------------------\\
(Vorname Name)
\end{flushright} \cleardoublepage
\section*{Abstract - deutsch}
Eine Zusammenfassung der Arbeit in deutscher Sprache.
\\\\
{\bf Stichworte:} experiment, kreativität, literatur review, creativity support system, geschäftsmodellmodellierungstool

\section*{Abstract - englisch}
Eine Zusammenfassung der Arbeit in englischer Sprache.
\\\\
{\bf Keywords:} experiment, creativity, design science research, literature review, creativity support system, businessmodelmodellingtool 
\\\\\\ \cleardoublepage
\pagestyle{scrheadings}
\frontmatter
\tableofcontents \cleardoublepage
%\addcontentsline{toc}{chapter}{\listfigurename}
\listoffigures \cleardoublepage
%\listofalgorithms \cleardoublepage
\listoftables \cleardoublepage
\addchap{Abkürzungsverzeichnis}
%\phantomsection \addcontentsline{toc}{chapter}{Abkürzungsverzeichnis}
%\renewcommand\refname{Abkürzungsverzeichnis} \chapter*{Abkürzungsverzeichnis}
\begin{acronym}[Bashgfgfdgfdgfd]
 \acro{KDE}{K Desktop Environment}
 \acro{SQL}{Structured Query Language}
 \acro{Bash}{Bourne-again shell}
 \acro{JDK}{Java Development Kit}
 \acro{VM}{Virtuelle Maschine}
 \acro{I2C}{Inter-Integrated Circuit}
\end{acronym} \cleardoublepage

%//Ausprobieren\\%
%\newfloat{ergebnisse}{thb}{loh}[chapter] % Liste der detaillierten Ergebnisse
%\floatname{ergebnisse}{Ergebnisse}
%\listof{ergebnisse}{Liste der detaillierten Ergebnisse}
%\cleardoublepage
%//Ausprobieren\\%

\mainmatter % der eigentliche Text folgt hier
\input{text/01einleitung}
%\input{text/stand}
%\input{text/konzept}
%\input{text/ergebnisse}
%\input{text/zusammenfassung}
\printbibliography
%\input{text/anhang} % Anhänge
\end{onehalfspace} %Ende 1,5-facher Zeilenabstand

\end{document}